 \section{Durchführung}
\label{sec:Durchführung}

\subsection{Vorbereitung}
\label{sec:Vorbereitung}

Für die periodischen Funktionen werden nun jeweils die Intervalle der ersten
Periode angegeben. Es gilt dann

\begin{equation}
  f(x) = f(x+2\pi n).
\end{equation}
Über die Formeln \eqref{eqn:anbn} werden
die Koeffizienten der jeweiligen Fourierreihe und ihre gesuchte
k-Anhängigkeit bestimmt.

\subsubsection{Rechteckspannung}

Eine Rechteckspannung wird beschrieben durch

\begin{equation}
  f_r(t) =
  \begin{cases}
    -1, & -\pi \leq t < 0 \\
    +1, & 0 \leq t < \pi
  \end{cases}.
\end{equation}
Daraus ergeben sich die k-Abhängigkeiten

\begin{equation}
  u_r(k) = 0
\end{equation}
für gerade k's und

\begin{equation}
  u_r(k) = \frac{1}{k}
\end{equation}
für ungerade k's.

\subsubsection{Sägezahnspannung}

Eine Sägezahnspannung wird beschrieben durch

\begin{align}
  f_s(t) & = t, & -\pi \leq t < \pi
\end{align}
Daraus ergeben sich die k-Abhängigkeiten

\begin{equation}
  u_s(k) = \frac{1}{k}
\end{equation}
für gerade und ungerade k's.

\subsubsection{Dreieckspannung}

Eine Dreieckspannung wird beschrieben durch

\begin{equation}
  f_d(t) =
  \begin{cases}
    \frac{\pi}{2} + t, & -\pi \geq t < 0 \\
    \frac{\pi}{2} - t, & 0 \leq t > \pi
  \end{cases}.
\end{equation}
Daraus ergibt sich die k-Abhängigkeiten

\begin{equation}
  u_d(k) = 0
\end{equation}
für gerade k's und

\begin{equation}
  u_d(k) = \frac{1}{k^2}
\end{equation}
für ungerade k's.

\subsection{Aufbau}

\subsubsection{Fourieranalyse}
\label{sec:AufbauA}

Für die Fourieranalyse wird ein Funktionsgenerator mit einem Oszilloskop
in Reihe geschaltet. Das Oszilloskop wird so eingestellt, dass das
Frequenzspektrum der benutzen Schwingung angezeigt wird.
Am Funktionsgenerator sollte eine möglichst hohe Grundfrequenz eingestellt
werden, damit die Abweichungen gering bleiben.

\subsubsection{Fouriersynthese}
\label{sec:AufbauS}

Bei der Fouriersynthese wird zur Einstellung der Amplitude ein
Signalgenerator mit einem Frequenzmessgerät parallel geschaltet. Zur Einstellung
der konstanten Phasenverschiebung von $\pi$ oder 0 wird das Frequenzmessgerät
mit dem Oszilloskop in Reihe geschaltet.
Um Lissajous-Figuren abzulesen muss der Oszillograph auf XY-Betrieb geschaltet
werden.


\subsection{Messvorgang}

\subsubsection{Fourieranalyse}

\begin{itemize}

  \item Zunächst wird die in \ref{sec:AufbauA} beschriebene Schaltung aufgebaut.

  \item Am Oszilloskop werden die ersten neun Frequenzpeaks aufgenommen.
  Um herauszufinden, ob es sich bei den kleineren Peaks um
  Nebenmaxima anstatt um eine gesuchte Eigenfrequenz handelt, kann der
  konstante Abstand zwischen den einzelnen Peaks ausgenutzt werden.
  Es werden Messwerte für eine Rechteck-, Dreieck- und Sägezahnspannung
  aufgenommen.

\end{itemize}

\subsubsection{Fouriersynthese}

Es werden die Graphen für eine Rechteck-, Dreieck- und Sägezahnspannung
aufgenommen. Dabei wird wie folgt vorgegangen:

\begin{itemize}

  \item Zunächst wird die in \ref{sec:AufbauS} beschriebene Schaltung zur
  Einstellung der Amplitude augebaut.

  \item Es wird am ersten Anschluss die maximale Amplitude eingestellt und
  abgelesen. Aus dem Messwert werden über die jeweilige k-Abhängigkeit aus
  \ref{sec:Vorbereitung} der
  Schwingung, die erzeugt werden soll, die Werte für $k=1,2,...,9$ berechnet.

  \item Nun wird die Amplitude des k-ten Anschluss' auf den berechneten k-ten
  Wert gesetzt.

  \item Es wird die in \ref{sec:AufbauS} beschriebene Schaltung zur Einstellung
  der Phasenverschiebung aufgebaut.

  \item Nun wird der k-te Anschluss mit dem ersten Anschluss eingeschaltet.
  Die Phasenverschiebung wird so verändert, dass bei ungeraden k's eine
  Lissajousfigur, die keine Fläche einschließt, und bei geraden k's
  eine Lissajousfigur, die symmetrisch zur x-Achse ist, erkennbar ist.

  \item Zuletzt wird das Oszilloskop zurückgesetzt und alle Anschlüsse
  eingeschaltet, sodass die jeweilige zu erzeugende Schwingung auf dem
  Bildschirm zu sehen ist.

  \item Es sollte darauf geachtet werden, dass immer nur die benötigten
  Schwingungen angeschlossen sind. Falls für bestimmte k's die berechnete
  Amplitude Null ist, kann der jeweilige Anschluss für alle Einstellungen
  vernachlässigt werden.

\end{itemize}
