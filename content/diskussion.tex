\section{Diskussion}
\label{sec:Diskussion}

\subsection{Fouriersynthese}

Mögliche Fehlerquellen bei allen drei Synthesen sind ungenau
eingestellte Amplituden und Phasenverschiebungen.

Die synthetisierte Rechteckspannung ist zwar klar zu erkennen,
weist aber noch Schwingungen bei den waagerechten Linien auf.
Diese Abweichung lässt sich durch die endliche Anzahl an zur Verfügung
stehenden Oberwellen erklären.
Eine perfekte Fouriersynthese lässt sich nur mit unendlich vielen
Oberwellen erreichen.

Die Sägezahnspannung weist in den schrägen Bereichen sogar noch größere Abweichungen
von der Idealform auf.

Die Dreiecksspannung kann gut realisiert werden. Eine Erklärung dafür
,dass sie im Vergleich zu den anderen beiden Spannungsarten so viel
besser dargestellt wird, ist, dass durch die $\frac{1}{k^2}$-Abhängigkeit
die Endlichkeit der Oberwellenanzahl weniger stark ins Gewicht fällt, da
höhere Oberwellen einen kleineren Beitrag leisten.

\subsection{Fourieranalyse}

Die gemessenen Amplituden aus dem Linienspektrum weichen bei der
Rechteckspannung stark vom theoretischen Wert ab. Die höchste
Abweichung ist \SI{93}{\percent} und die niedrigste \SI{43}{\percent}.
Eine Erklärung für die hohe Abweichung auch im Vergleich zu den
anderen Schwingungsarten konnte nicht gefunden werden.

Bei der Sägezahnspannung sind die Werte erfreulich nah am Theoriewert.

Die Abweichungen der Werte der Dreiecksspannung sind größtenteils
dem schwierigen Ablesen geschuldet, da sie sehr klein sind und leicht
von Störfaktoren überlagert werden können.
